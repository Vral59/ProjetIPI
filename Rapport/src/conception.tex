\renewcommand{\chaptername}{Partie}

\ifsolo
    ~

    \vspace{1cm}

    \begin{center}
        \textbf{\LARGE Les améliorations} \\[1em]
    \end{center}
    \tableofcontents
\else
    \chapter{Les améliorations}

    \minitoc
\fi
\thispagestyle{empty}

\ifsolo \newpage \setcounter{page}{1} \fi


\section{Améliorations possibles}
\subsection{Gestion des erreurs}
Ne sachant pas comment gérer les erreurs dans un projet un peu plus professionnel, j'ai garder l'habitude de mettre des \texttt{exit(1)}. Une plus grande diversité et description des erreurs pourrait être utile à corriger le code.
\subsection{Makefile}
Je pense que le  \texttt{Makefile} pourrait être amélioré, même s'il fonctionne. Il n'a surement pas une forme conventionnelle et il ne supprime pas automatiquement les fichiers en \texttt{*.o}
\subsection{Bonus}
Par manque de temps et de compréhension de la bibliothèque \texttt{DOT}, je n'ai pas pu travailler sur l'aspect graphique de l'automate, cela pourrait être fait plus tard.
\subsection{Les commentaires}
J'ai eu quelques difficultés à savoir quels commentaires mettre et à quel emplacement. De plus, il est difficile d'être concis et précis à la fois. En effet certaines lignes qui me semblent claires peuvent ne pas l'être pour d'autres, et inversement.
\subsection{Notes supplémentaires}
Mon projet fonctionne apriori correctement sur les 4 exemples proposés. Cependant je n'ai pu le tester dans des conditions particulières. Par exemple, un automate qui peut contenir plus de 173 caractères, peut-être un soucis si je n'arrive pas à bien faire la distinction entre les trois '\textbackslash255' qui séparent \texttt{decale} et \texttt{branchement} dans le fichier et l'état 173 (car le numéro ascii de \textbackslash255 est 173).\\
Il aurait peut être été préférable de réduire la quantité de code dans le \texttt{main} et de faire un autre fichier pour contenir toutes les lignes qui servent à la lecture et au stockage de l'automate. Ici ce n'est pas encore trop gênant car la taille \texttt{main} reste raisonnable, c'est pour cela que je ne l'ai pas fait.